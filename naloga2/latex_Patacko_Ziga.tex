\documentclass[11pt,a4paper]{article}

\usepackage[slovene]{babel}
\usepackage{color}
\usepackage{subfigure}
\usepackage{verbatim}
\usepackage{amsmath}

\begin{document}

\title{Kratka doma\v ca vaja iz \LaTeX-a}
\author{\v Ziga Pata\v cko Koderman}
\date{\today}

\maketitle
\section{Besedilo}
{\bf Navldilo:}
\begin{enumerate}
	\item v  Latex  prepi\v site  vsebino  \underline{celotnega}
dokumenta. Pazite na to\v cko 3c. Sklicujemo se tudi na \v clanek \cite{clanek} in ena\v cbo \ref{eq:1} v poglavju \ref{sec:izrazi}. Vsa sklicevanja naj bodo v izvorni kodi dejansko sklicevanja (\verb_\ref{}_ in \verb_\cite{}_).
	\item V naslovu ne pozabite mojega imena zamenjati s svojim.
	\item V enem samem emailu boste poslali:
	\begin{enumerate}
		\item PDF datoteko, generirano iz va\v se izvorne Latex kode
		\item tekstovno datoteko z vaˇso izvorno Latex kodo, ki se lepo prevede
		\item ime obeh datotek naj bo:  latex\_Priimek\_Ime.*
	\end{enumerate}
\end{enumerate}

\section{Matemati\v cni izrazi}\label{sec:izrazi}
Prvi izraz:
\begin{equation}\label{eq:1}
F_{\rho n} = \frac{1}{2} \int \!\!\!\! \int dz\ {\rm d}^2 r_\bot \Bigg[ B \bigg( \frac{\delta \rho}{\rho_0} \bigg)^2 + B'\bigg( \frac{ |\nabla \delta \rho| }{\rho_0} \bigg)^2 + K_1(\nabla_\bot \cdot \delta  \boldsymbol{{\rm n}})^2 + K_3 |\partial_z \delta \boldsymbol{{\rm n}}|^2 \Bigg]
\end{equation}
Drugi izraz:
$$
S^t (\boldsymbol{{\rm q}}) = k_B T \rho_0 \frac
{(3 s_0 q_z q_\bot)^2 + (K_1 q^2_\bot + K_3q^2_z) / \tilde{H}}
{\tilde{B} (3 s_0 q_z q_\bot)^2 + \bigg\{ \tilde{B} / \tilde{H} + \big[ (s_0 + \frac{1}{2})q_z^2 + \frac{1}{2}(1-s_0)q_\bot^2 \big]^2 \bigg\} (K_1q_\bot^2 + K_3q_z^2) } .
$$

\begin{thebibliography}{99}
	\bibitem{clanek} A. Avtor, B. Bvtor, Rev. Neki. Phys. {\bf 42}, 685 (2015).
\end{thebibliography}

\end{document}
